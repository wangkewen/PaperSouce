\section{Related Work}
\label{related}

\noindent
While significant volume of work exists that looked into various aspects of performance prediction in various distributed settings, in this paper, we focus our discussion on recent efforts that investigated the challenge of performance prediction for various cloud platforms~\cite{predict, nosqlmodel, pmodel, starfish, oltp, prepare, cloudopt, cloudscale, dbseer, amml}. Among numerous efforts, PREDIcT~\cite{predict} is one of the recent work that looked into the problem of predicting runtime for network intensive iterative algorithms implemented on Hadoop MapReduce platform. Specifically, it aims to predict the number of iterations and runtime for each iteration based on sample run and historical data. However, PREDIcT focuses on iterative algorithms, and requires representative training dataset to achieve high prediction accuracy, and may lead to poor prediction accuracy for applications with no historical data. To simplify the performance prediction for complex Hadoop application, another recent work presented the idea of using a modeling language (e.g., Hive Query Language (HQL))~\cite{nosqlmodel} that translates big data applications into SQL-like queries on Apache Hive \cite{hive}. This provides a convenient way for predicting performance for data processing applications that can be implemented using HQL queries. For Map Reduce jobs running on heterogeneous machines, bound-based performance modeling techniques are tried for predicting job completion time~\cite{pmodel} as well. The main idea is to evaluate the upper and lower bounds of job completion time, and use that to predict job performance. Starfish~\cite{starfish} presents a self-tuning framework for MapReduce paradigm to predict job performance under different program configurations for Apache Hadoop jobs. It applies analytical approach to estimate how a job will perform based on job simulation data, and uses that model to predict performance. While this and prior approaches achieve good prediction accuracy, due to the differences between the implementation of other cloud platforms and Apache Spark platform, Starfish and similar approaches are not suitable for predicting job performance running on Apache Spark platform without significant modifications. We address this void in our paper as follows.

