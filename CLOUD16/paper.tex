


\newcommand{\nop}[1]{}
\documentclass[conference]{IEEEtran}

\usepackage{amsmath}
\usepackage[belowskip=-6pt,aboveskip=0pt]{caption}
\usepackage[lined,linesnumbered,noresetcount]{algorithm2e}
\usepackage{amssymb}
\setcounter{tocdepth}{3}
\usepackage{graphicx}
\usepackage{tablefootnote}
\usepackage{url}
  
\usepackage{caption}
\usepackage{mathtools}
\DeclarePairedDelimiter\floor{\lfloor}{\rfloor}









\ifCLASSINFOpdf
 
\else
 
\fi






\hyphenation{op-tical net-works semi-conduc-tor}


\begin{document}


\title{Modeling Interference for Apache Spark Jobs}



\author{\IEEEauthorblockN{Kewen Wang, Mohammad Maifi Hasan Khan, Nhan Nguyen and Swapna Gokhale}
\IEEEauthorblockA{Department of Computer Science and Engineering\\
University of Connecticut\\ Email: kewen.wang@uconn.edu, maifi.khan@engr.uconn.edu,nhan.q.nguyen@uconn.edu,ssg@engr.uconn.edu}}


\maketitle


\begin{abstract}

To maximize resource utilization and system throughput, hardware resources are often shared across multiple Apache Spark jobs through virtualization techniques in cloud platforms. However, while the performance of these jobs running in virtualized environment can be negatively affected due to interference caused by resource contention, it is nontrivial to predict the effect of interference on job performance in such settings, which is critical for efficient scheduling of such jobs and performance troubleshooting. To address this challenge, in this paper, we develop analytical models to estimate the effect of interference among multiple Apache Spark jobs running concurrently on job execution time in virtualized cloud environment. We evaluated the accuracy of our models using four real-life applications (e.g., Page rank, K-means, Logistic regression, and Word count) on a 6 node cluster while running up to four jobs concurrently. Our experimental results show that the model can achieve high prediction accuracy, and ranges between 86\% to 99\% when the number of concurrent jobs are four and all start simultaneously, and ranges between 71\% to 99\% when the number of concurrent jobs are four and start at different times.


 



\end{abstract}

\begin{IEEEkeywords}

Apache Spark; Modeling Performance Interference; Execution Time Prediction

\end{IEEEkeywords}


\IEEEpeerreviewmaketitle

\section{Introduction}
\noindent
Among many cloud computing platforms, Apache Spark~\cite{spark} is one of the popular open-source cloud platforms that introduces the concept of resilient distributed datasets (RDDs)~\cite{rdd} to enable fast processing of large volume of data leveraging distributed memory. Its in-memory data operations makes it well-suited for iterative applications such as iterative machine learning and graph algorithms. However, execution time of a particular job on Apache Spark platform can vary significantly depending on the input data type and size, design and implementation of the algorithm, and computing capability (e.g., number of nodes, CPU speed, memory size), making it extremely difficult to predict job performance, which is often needed to optimize resource allocation\cite{cloudopt}\cite{cloudscale}. Performance prediction can also help to locate execution stages with abnormal resource usage pattern~\cite{prepare}. 
While prior work exists that looked into the problem of performance prediction for cloud platforms such as Apache Hadoop \cite{hadoop} (an open-source implementation of MapReduce \cite{mr} computing framework), these approaches are not suitable for apache Spark platform due to its different programming model and features such as in-memory data operations.
\noindent
Hence, to address this void, in this paper, we focus on performance modeling for Apache Spark jobs. While various forms of machine learning approaches are often used to predict system performance leveraging past system execution data~\cite{stat}\cite{predict}\cite{oltp} and can achieve reasonable prediction accuracy, it requires training dataset. In contrast, modeling based approaches predict performance through modeling system behavior~\cite{starfish}\cite{nosqlmodel}\cite{pmodel}, and often can provide a better understanding regarding internal execution of a program and resulting performance. Therefore, in this paper, we apply analytical approaches to predict the performance of Apache Spark jobs. Specifically, we leverage the multi-stage execution structure of Apache Spark jobs to develop hierarchical models that can effectively capture the execution behavior of different execution stages. Using these models, we first measure the job performance based on limited scale execution using only a fraction of real data set. Next, we predict the job performance based on the limited scale execution job performance data. We evaluated our framework with four real-world applications. In each case, our model is able to predict execution time for individual stage with high accuracy. Additionally, the model is able to predict memory requirement for RDD creation with high accuracy. However, the accuracy of I/O cost prediction varied for different applications and simulation setup. We discuss our detailed findings in Section~\ref{evaluation}.

\noindent
The rest of the paper is organized as follows. Prior efforts that are related to our work is discussed in Section~\ref{related}. Section~\ref{overview} explains the models that are used to predict job performance in this work. Experimental results are presented in Section~\ref{evaluation}. Finally, Section~\ref{conclusion} concludes the paper.


\section{Related Work}
\label{related}

\noindent
While significant volume of work exists that looked into various aspects of performance prediction in various distributed settings, in this paper, we focus our discussion on recent efforts that investigated the challenge of performance prediction for various cloud platforms~\cite{predict, nosqlmodel, pmodel, starfish, oltp, prepare, cloudopt, cloudscale, dbseer, amml}. Among numerous efforts, PREDIcT~\cite{predict} is one of the recent work that looked into the problem of predicting runtime for network intensive iterative algorithms implemented on Hadoop MapReduce platform. Specifically, it aims to predict the number of iterations and runtime for each iteration based on sample run and historical data. However, PREDIcT focuses on iterative algorithms, and requires representative training dataset to achieve high prediction accuracy, and may lead to poor prediction accuracy for applications with no historical data. To simplify the performance prediction for complex Hadoop application, another recent work presented the idea of using a modeling language (e.g., Hive Query Language (HQL))~\cite{nosqlmodel} that translates big data applications into SQL-like queries on Apache Hive \cite{hive}. This provides a convenient way for predicting performance for data processing applications that can be implemented using HQL queries. For Map Reduce jobs running on heterogeneous machines, bound-based performance modeling techniques are tried for predicting job completion time~\cite{pmodel} as well. The main idea is to evaluate the upper and lower bounds of job completion time, and use that to predict job performance. Starfish~\cite{starfish} presents a self-tuning framework for MapReduce paradigm to predict job performance under different program configurations for Apache Hadoop jobs. It applies analytical approach to estimate how a job will perform based on job simulation data, and uses that model to predict performance. While this and prior approaches achieve good prediction accuracy, due to the differences between the implementation of other cloud platforms and Apache Spark platform, Starfish and similar approaches are not suitable for predicting job performance running on Apache Spark platform without significant modifications. We address this void in our paper as follows.


\section{Overview}
\label{overview}

\noindent
In Apache Spark platform, each job consists of multiple execution stages implementing distinct operations of an application program where stages are executed sequentially. To facilitate parallel processing, input data set is partitioned into multiple sets and are distributed over multiple worker nodes. Within each worker node, batches of tasks are launched to process the corresponding partition of the input data. The number of tasks within each node is determined based on the size of the input data and configuration settings of the program. 
\noindent
To illustrate the main idea behind Apache Spark job execution, let us consider the Apache Spark PageRank job running on two worker nodes: A and B, where Node A has 8 CPU cores and Node B has 12 CPU cores as shown in Figure~\ref{flow}. This PageRank job will have 13 stages if the iteration number is set to 10, where stage 1 and stage 2 execute the distinct() operation. In the iteration part from stage 3 to stage 12, the operation reduceByKey() is executed. The final stage performs the saveAsTextFile() operation. As shown in Figure~\ref{flow}, each box in the Figure represents one stage, and each line in the box represents one task. Different colors are used to differentiate tasks running on different worker nodes. If the input data size of this PageRank job is 2.5 GB, the total number of input blocks will be 40 for a default block size of 64 MB. As the number of tasks is same as the input block number, there are 40 lines in each stage. In addition, the number of tasks in each stage is same within one Spark job. Therefore, for this example, in each stage, 40 tasks will be executed. However, different CPU core may complete different number of tasks due to the difference in computing ability and uncertainty during the program execution. 
\noindent
Given the above model of execution, next, we present the developed hierarchical models that can be used to predict job execution time, memory footprint for RDD creation, and I/O overhead as follows. 


\subsection{Model for Estimating Execution Time}

\noindent
As a Spark job is executed in multiple stages where each stage contains multiple tasks, we use the following notation to represent an Apache Spark job:
\begin{IEEEeqnarray}{rCl}
\label{jobperform}
Job &{} ={}& \{Stage_i \mid 0 \leq i \leq M \} \IEEEyessubnumber\\
Stage_i &{} ={}& \{Task_{i,j} \mid 0 \leq j \leq N \} \IEEEyessubnumber%
\end{IEEEeqnarray}
Here $M$ is the number of stages in a job and $N$ is the number of tasks in a stage. 
\begin{figure}[!t]
\centering
\includegraphics[width=3.0in]{figures/flow.png}
\caption{Prediction Accuracy for WordCount}
\label{flow}
\end{figure}
\noindent
Next, as different stages within a job are executed sequentially, we represent the execution time of a job as the sum of the execution time of each stage plus the job startup time and the job cleanup time as follows:
\begin{IEEEeqnarray}{rCl}
\label{jobtime}
JobTime = Startup + \sum_{s=1}^{M} StageTime_{s} + Cleanup
\end{IEEEeqnarray}
Next, within each stage, as one CPU core executes one task at a time, in a cluster with $H$ worker nodes, the number of tasks $P$ that can run in parallel can be calculated as follows:
\begin{IEEEeqnarray}{rCl}
\label{paralltask}
P=\sum_{i=1}^H CoreNum_{i}
\end{IEEEeqnarray}
Here, $CoreNum_{i}$ is the number of CPU cores of working node $i$ and $H$ is the number of working nodes in the cluster. Hence, within an execution stage, tasks in each stage are executed in batches where each batch consists of $P$ tasks running in parallel. However, due to the differences in computing capabilities among different worker nodes in a heterogeneous cluster and inherent uncertainty in program execution, the execution time of different tasks may vary significantly. Therefore, the time spent in a particular stage can be calculated as the maximum of the sum of all the sequential tasks' time within a stage plus the stage startup time and the stage cleanup time as follows:
\begin{IEEEeqnarray}{rCl}
\label{stagetime}
StageTime&{} ={}& Startup + \max_{c=1}^{P} \sum_{i=1}^{K_c} TaskTime_{c,i} \nonumber \\
&&+Cleanup
\end{IEEEeqnarray}
Here $P$ is the number of total CPU cores, and $K_c$ is the number of sequential tasks executed on CPU core $c$.
Finally, as different tasks in a stage follow the same execution pattern, the execution time of a task can be computed as follows:
\begin{IEEEeqnarray}{rCl}
\label{tasktime}
TaskTime&{} ={}& DeserializationTime + RunTime \nonumber \\
&&+ SerializationTime
\end{IEEEeqnarray}
Here $DeserializationTime$ is the time taken to deserialize the input data, $SerializationTime$ is the time taken to serialize the result, and $RunTime$ is the actual time spent performing operations on data such as data mapping, filtering, calculating, and analyzing. 


\subsection{Memory Consumption}

\noindent
As the Spark platform takes the advantage of in-memory processing to improve the computing efficiency, it is important to allocate sufficient memory needed to create initial RDD to avoid possible slowdown of the execution. Moreover, under certain system configurations, lack of enough memory for initial RDD creation can lead to unexpected program termination. To avoid such adverse events, we develop a simple model to estimate the minimum memory requirement for RDD creation. Specifically, if there are N tasks in the system, we can express the total memory requirement for the job as follows:
\begin{IEEEeqnarray}{rCl}
\label{jobmem}
JobRddMem=\sum_{i=1}^{N} TaskRddMem_{i} 
\end{IEEEeqnarray}





\subsection{Model for Estimating I/O Cost}

\noindent
Finally, within a stage, the transformation operation that generates new RDD based on previous RDD is implemented in $ShuffleMapTask$ and the action operation that output the result data which is implemented in $ResultTask$. The I/O cost during the shuffle phase in these two types of tasks can be classified into two categories, namely, the shuffle read cost and the shuffle write cost. Shuffle write cost is the cost of writing the interim data to local disk buffer, and shuffle read cost refers to the network I/O cost for fetching the interim data from other worker nodes. As shuffle phase is the most I/O intensive phase where frequent data fetching and transmission occurs, in our model, for I/O cost measurement, we specifically focus on data transmission during the shuffle phase that involves data fetching from remote hosts and the interim data writing into the disk. The stage-by-stage I/O cost is calculated as follows:
\begin{IEEEeqnarray}{rCl}
\label{stagewriteio}
StageIOWrite_{i} &{} ={}& \sum_{j=1}^{N} TaskIOWrite_{i,j} \IEEEyessubnumber\\
\label{stagereadio}
StageIORead_{i} &{} ={}& \sum_{j=1}^{N} TaskIORead_{i,j} \IEEEyessubnumber%
\end{IEEEeqnarray}
Here $N$ is the number of tasks in $Stage_i$.



\subsection{ Performance Prediction}
\label{predictsec}

\noindent
Based on the above model, to predict job performance, the presented modeling framework first executes the program on a cluster using limited amount of sample input data and collect performance metrics such as run time, I/O cost, and memory cost during the simulated run. Next, the extracted performance metric from simulated run is used to predict the performance metric for the actual run. 
\noindent
Specifically, to predict the execution time, we first calculate the number of tasks that will be executed in the actual job as follows: $ N = InputSize / BlockSize $, where $InputSize$ is the size of the input data, and $Blocksize$ is the size of one data block in HDFS. 
The tasks within a stage are scheduled to run batch by batch, and the number of tasks in each batch $P$ is computed as shown in equation (\ref{paralltask}). In one batch of tasks, while the tasks may start simultaneously, they may not finish at the same time due to various factors such as data skew problem, and differences in computing capability of different worker nodes. Hence, using simulation data, we calculate the average execution time for a task for a given stage for a worker node $h$ as follows. 
\begin{IEEEeqnarray}{rCl}
\label{taskest}
TaskRunTime_{h,i} &{} ={}& DeserializeTime_{h,i} \nonumber \\
&&+ RunTime_{h,i} \nonumber \\
&&+ SerializationTime_{h,i} \\
\label{avgtask}
AvgTaskTime_h &{} ={}& \frac{1}{n_h}\sum_{i=1}^{n_h} TaskRunTime_{h,i}
\end{IEEEeqnarray}
Here $n_h$ is the number of tasks running in host $h$ in a particular stage of the sample job. 
Moreover, during our experiment, we observed that the average execution time of the first batch is significantly different compared to the subsequent batches within the same stage, which we capture as follows. 
\begin{IEEEeqnarray}{rCl}
\label{ratio}
Ratio_h=\frac{\frac{1}{n_h-P_h} \sum_{i= P_h + 1}^{n_h}TaskTime_{h,i}}{\frac{1}{P_h} \sum_{j=1}^{P_h} TaskTime_{h,j}}
\end{IEEEeqnarray}
Here $n_h$ is the number of tasks running in host $h$, and $P_h$ is the number of tasks in the first batch. Please note that, to trace two batches of tasks to calculate this ratio for every working node, $SampleSize$ needs to be doubled (discussed in Section~\ref{sampling}).
As tasks execute on different hosts in parallel, to predict the execution time for a particular stage during actual execution, stage $Startup$ time and $Cleanup$ time are viewed as constants which are extracted from simulation logs, and stage execution time is estimated as follows: 
\begin{IEEEeqnarray}{rCl}
\label{stageest}
EstStageTime&{} ={}&Startup + \max_{c=1}^{P} \sum_{i=1}^{K_c} AvgTaskTime_{c,i} \nonumber \\
&&+ Cleanup \\
\label{stagetask}
EstTaskTime_{c,i}&{} ={}&\left\{\begin{IEEEeqnarraybox}[\relax][c]{l's}
AvgTaskTime_c,& $i = 1$\\
AvgLaterTaskTime_c,& $i > 1$%
\end{IEEEeqnarraybox}\right.
\end{IEEEeqnarray}
Here $P$ is the number of total CPU cores calculated in equation (\ref{paralltask}), $K_c$ is the number of sequential tasks running in CPU core $c$. $AvgTaskTime_c$ is the average time for tasks in the first batch for CPU core $c$ of the corresponding host, and is calculated in equation (\ref{avgtask}). $AvgLaterTaskTime_c$ is the average time of the following batches of tasks, which could be calculated as $Ratio_h \times AvgTaskTime_h$.
For predicting I/O cost, the average shuffle read and write costs of a typical task is computed and then used to compute the I/O cost for a specific stage j as follows:
\begin{IEEEeqnarray}{lCl}
\label{stageio}
EstStageIOWrite_j \nonumber \\
=\sum_{h=1}^{H} (N_h \times \frac{1}{n_h} \sum_{i=1}^{n_h} (TaskIOWrite_{h,i}))
\end{IEEEeqnarray}
\begin{IEEEeqnarray}{lCl}
\label{stageio}
EstStageIORead_j \nonumber \\
=\sum_{h=1}^{H} (N_h \times \frac{1}{n_h} \sum_{i=1}^{n_h} (TaskIORead_{h,i}))
\end{IEEEeqnarray}
Here $H$ is the number of worker nodes and $N_h$ is the number of total tasks on host h during real execution. $n_h$ is the number of tasks running on host $h$ during simulation at stage j. 
\noindent
Finally, the average memory footprint for each stage is estimated as follows: 
\begin{IEEEeqnarray}{lCl}
\label{stagemem}
EstRddMem = \sum_{h=1}^{H} ( \frac{N_h}{n_h} \sum_{i=1}^{n_h} TaskRddMem_{h,i} )
\end{IEEEeqnarray}


\subsection{Simulation Methodology }
\label{sampling}

\noindent
For simulation, we tried two alternative setup as follows. 
In the first setup, to make sure that all worker nodes in the cluster is used during simulation, we extract sufficient amount of sample input data so that each CPU core gets to process at least one block of input data. Hence, given that one block of HDFS data is configured to be equal to the size $BlockSize$, the minimum value of $SampleSize$ can be calculated as $BlockSize \times P$, where $P$ is the number of tasks that can run in parallel ($P$ is calculated in equation (\ref{paralltask})). 
\noindent
However, as the prediction mechanism needs to extract $2 \times P$ blocks of sample data from the original input to simulate on the whole cluster, for clusters with a large number of worker nodes, total CPU core number $P$ may be very huge, resulting in a big sample job and long simulation time. In order to reduce the simulation time, we tried another alternative where the sample job is executed in a smaller cluster which has fewer number of CPU cores $p$. As a result, only $2 \times p$ blocks of sample data is needed. In our simulation, we ran simulation on a smaller cluster that includes one node of each type (as shown in Figure~\ref{fig:cluster}(b)). Basically, all computing nodes in a cluster can be classified into $D$ groups, where each group has $Num_g$ computing nodes and each computing node in a group has the same computing capability (e.g., CPU speed, RAM). Next, one node is selected from each group, and total $D$ working nodes are chosen to construct this new cluster. In such a setting, the size of sample data is reduced to $D / \sum_{g=1}^{D} Num_g$ times of the original input data, reducing the simulation time significantly. 
Finally, to reduce the impact due to data skew, our sampling technique divides the input data into multiple sections, and extracts data from each section with equal probability.
\begin{figure}[!t]
\centering
\includegraphics[width=3.0in]{figures/cluster.png}
\caption{Simulation Setup}
\label{fig:cluster}
\end{figure}


\section{Evaluation}
\label{evaluation}
\noindent
To evaluate the accuracy of our modeling framework, we used a cluster of 6 nodes and used Xen hypervisor \cite{xen} to create up to four virtual machines on each physical machine. Each virtual machine is configured with 4GB of memory and 1 CPU core. For the deployed Apache Spark platform, one machine serves as the master node, and the remaining five machines serve as working nodes. In six physical machines, we create multiple clusters leveraging virtual machines to execute multiple Apache Spark jobs in parallel. 
\noindent
In our evaluation, for prediction, first, we need to estimate the parameter $\beta_{n}$ in equation (\ref{beta}). Towards that, we implemented our own Apache Spark job and executed that on our cluster to obtain the execution time and resource consumption information. This simulation job consists of three stages executing distinct(), groupByKey(), and count() operation respectively. Distinct() implements a mapping function and parses the input data, groupByKey() processes the output of distinct() operation, and count() is a CPU intensive operation performing data summarization. This simulation job is executed with 2.5 GB of sample data where the first stage implementing the Distinct() operation involves significant I/O compared to the following two stages. To measure $\beta_{n}$, we executed $n$ ($n$=1,2,3,4) instances of this simulation job in parallel. As shown in Figure \ref{simjobs}, the effect of interference is significant for the first stage but minimal for the subsequent stages. 
\begin{figure}[!t]
\centering
\captionsetup{justification=centering}
\includegraphics[width=3in]{figures/simjobs.png}
\caption{Execution Time for Different Number of Simulation Jobs}
\label{simjobs}
\end{figure}
\noindent
Once we estimate the value of $\beta_{n}$, subsequently, we used our formulation to predict the execution time for each stage for each job separately in different execution scenarios and add up the prediction error for each stage to calculate the total prediction accuracy $R$ as below.
\begin{equation}
\label{predictaccuracy}
R = |1 - \sum_{i=1}^{M} \frac{|PredictedTime_i - MeasuredTime_i|}{\sum_{j=1}^{M} Time_{j}}|
\end{equation}
\noindent
Here $M$ is the number of stages in a job, $PredictedTime_i$ is the predicted execution time for $stage_i$, and $MeasuredTime_i$ is the actual execution time of $stage_i$. Different evaluation scenarios are presented below.



\subsection{Interference Among Multiple Jobs of the Same Type Starting Simultaneously}
\noindent
In this part of the evaluation, we present the accuracy of prediction while modeling the effect of interference among multiple jobs of the same type (e.g., interference between n instances of job x). Towards that, we choose four Apache Spark jobs: PageRank, K-Means, Logistic Regression and WordCount. WordCount job is a non-iterative job while the remaining three are iterative jobs. For PageRank, we use the LiveJournal network dataset from SNAP \cite{snap}, which is processed through mapping each node id into longer string to form a 20 GB input data set. K-Means and Logistic Regression applications use 20 GB of numerical Color-Magnitude Diagram data of galaxy from Sloan Digital Sky Survey (SDSS) \cite{sdss}. WordCount application uses 20 GB of Wikipedia dump data.
\begin{table}[!t]
\renewcommand{\arraystretch}{1.3}
\caption{Prediction Accuracy for Interference Among Same Jobs}
\label{table_samejob}
\centering
\begin{tabular}{c|c|c|c}
\hline
\bfseries JobName & \bfseries Job Number & \bfseries First Stage & \bfseries Whole Job\\
\hline\hline
PR & 2 & 0.97 & 0.80\\
& 3 & 0.96 & 0.85\\
& 4 & 0.92 & 0.82\\
\hline
KM & 2 & 0.75 & 0.70\\
& 3 & 0.71 & 0.68\\
& 4 & 0.98 & 0.92\\
\hline
LR & 2 & 0.74 & 0.78\\
& 3 & 0.79 & 0.81\\
& 4 & 0.97 & 0.97 \\
\hline
WC & 2 & 0.87 & 0.86\\
& 3 & 0.96 & 0.94\\
& 4 & 0.95 & 0.94\\
\hline
\end{tabular}
\end{table}
\noindent
For prediction, we first executed the sample job (e.g., Page rank) with 2.5 GB of input data to collect the job execution profile, which is then used to predict the execution time assuming no interference. Finally, we used our framework to adjust the prediction assuming interference. The prediction accuracy is summarized in Table~\ref{table_samejob}. In the table, PR, KM, LR, and WC refers to PageRank, K-Means, Logistic Regression, and WordCount application respectively. Column Job number (e.g., 2, 3, 4) indicates the number of jobs that were executed in parallel. For instance, a value of 2 indicates that two instances of the same job were executed in parallel. As can be seen, prediction accuracy is highest for Logistic regression application (97\%) and lowest for K-means (68\%).The predicted execution time and the actual execution time when we executed four instances of the same job in parallel are shown in Figure~\ref{pr20gprIII}, Figure~\ref{km20gkmIII}, Figure~\ref{lr20glrIII}, and Figure~\ref{wc20gwcIII} for PageRank, K-Means, Logistic Regression, and WordCount respectively. 
\begin{figure}[!t]
\centering
\captionsetup{justification=centering}
\includegraphics[width=3in]{figures/pr20g_prIII.png}
\caption{Execution Time Prediction for Four Interfered PageRank Jobs}
\label{pr20gprIII}
\end{figure}
\begin{figure}[!t]
\centering
\captionsetup{justification=centering}
\includegraphics[width=3in]{figures/km20g_kmIII.png}
\caption{Execution Time Prediction for Four Interfered K-Means Jobs}
\label{km20gkmIII}
\end{figure}
\begin{figure}[!t]
\centering
\captionsetup{justification=centering}
\includegraphics[width=3in]{figures/lr20g_lrIII.png}
\caption{Execution Time Prediction for Four Interfered Logistic Regression Jobs}
\label{lr20glrIII}
\end{figure}
\begin{figure}[!t]
\centering
\captionsetup{justification=centering}
\includegraphics[width=3in]{figures/wc20g_wcIII.png}
\caption{Execution Time Prediction for Four Interfered WordCount Jobs}
\label{wc20gwcIII}
\end{figure}
\noindent


\subsection{Interference Among Multiple Jobs of Different Types Starting Simultaneously}
\noindent
In this section, we present the accuracy of prediction while modeling the interference among $n$ different jobs concurrently, where $n$ was varied between 2 to 4. For example, when $n$ = 2, we execute two different jobs concurrently. The prediction accuracy while running two different jobs in parallel is summarized in Table~\ref{table_twojobs}. As shown in the table, there are a total of 6 combinations to consider. As can be seen, prediction accuracy ranges between 97\% and 69\% for the whole job, and between 99\% and 70\% for the first stage, which incurs the bulk of the execution time. 
\noindent
For $n$=3, we execute three different jobs concurrently. The prediction accuracy while running three different jobs in parallel is summarized in Table~\ref{table_threejobs}. As shown in the table, there are a total of 4 combinations to consider. As can be seen, prediction accuracy ranges between 90\% and 79\% for the whole job, and between 99\% and 83\% for the first stage.
\noindent
Finally, for $n$=4, we execute four different jobs concurrently. The prediction accuracy while running four different jobs in parallel is summarized in Table~\ref{table_fourjobs}. As shown in the table, there are a total of 1 combination to consider. As can be seen, prediction accuracy ranges between 99\% and 86\% for the whole job, and between 99\% and 92\% for the first stage.
\begin{table}[!t]
\renewcommand{\arraystretch}{1.3}
\caption{Prediction Accuracy for Two Different Jobs}
\label{table_twojobs}
\centering
\begin{tabular}{c|c|c|c}
\hline
\bfseries JobName & \bfseries Interfered Job & \bfseries First Stage & \bfseries Whole Job\\
\hline\hline
PR & KM & 0.91 & 0.79\\
& LR & 0.93 & 0.81\\
& WC & 0.99 & 0.85\\
\hline
KM & PR & 0.89 & 0.80\\
& LR & 0.80 & 0.73\\
& WC & 0.75 & 0.69\\
\hline
LR & PR & 0.97 & 0.97 \\
& KM & 0.73 & 0.77\\
& WC & 0.70 & 0.75\\ 
\hline
WC & PR & 0.96 & 0.87\\
& KM & 0.93 & 0.84\\
& LR & 0.96 & 0.88\\
\hline
\end{tabular}
\end{table}
\begin{table}[!t]
\renewcommand{\arraystretch}{1.3}
\caption{Prediction Accuracy for Three Different Jobs}
\label{table_threejobs}
\centering
\begin{tabular}{c|c|c|c}
\hline
\bfseries JobName & \bfseries Interfered Jobs & \bfseries First Stage & \bfseries Whole Job\\
\hline\hline
PR & KM, LR & 0.96 & 0.87\\
& KM, WC & 0.99 & 0.90\\
& LR, WC & 0.99 & 0.90\\
\hline
KM & PR, LR & 0.84 & 0.79\\
& PR, WC & 0.92 & 0.87\\
& LR, WC & 0.83 & 0.80\\
\hline
LR & PR, KM & 0.84 & 0.85\\
& PR, WC & 0.87 & 0.88\\
& KM, WC & 0.83 & 0.84\\
\hline
WC & PR, LR & 0.93 & 0.87\\
& PR, KM & 0.93 & 0.87\\
& KM, LR & 0.94 & 0.89\\
\hline
\end{tabular}
\end{table}
\begin{table}[!t]
\renewcommand{\arraystretch}{1.3}
\caption{Prediction Accuracy for Four Different Jobs}
\label{table_fourjobs}
\centering
\begin{tabular}{c|c|c|c}
\hline
\bfseries JobName & \bfseries Interfered Jobs & \bfseries First Stage & \bfseries Whole Job\\
\hline\hline
PR & KM, LR, WC & 0.92 & 0.86\\
\hline
KM & PR, LR, WC & 0.99 & 0.95\\
\hline
LR & PR, KM, WC & 0.99 & 0.99 \\
\hline
WC & PR, KM, LR & 0.95 & 0.90\\
\hline
\end{tabular}
\end{table}
The predicted execution time and the actual execution time when we executed four different jobs in parallel are shown in Figure~\ref{pr20gkmlrwc}, Figure~\ref{km20gprlrwc}, Figure~\ref{lr20gprkmwc}, and Figure~\ref{wc20gprkmlr} for PageRank, K-Means, Logistic Regression, and WordCount respectively. 
\begin{figure}[!t]
\centering
\captionsetup{justification=centering}
\includegraphics[width=3in]{figures/pr20g_km_lr_wc.png}
\caption{Execution Time Prediction for PageRank Job Interfered with Other Three Jobs}
\label{pr20gkmlrwc}
\end{figure}
\begin{figure}[!t]
\centering
\captionsetup{justification=centering}
\includegraphics[width=3in]{figures/km20g_pr_lr_wc.png}
\caption{Execution Time Prediction for K-Means Job Interfered with Other Three Jobs}
\label{km20gprlrwc}
\end{figure}
\begin{figure}[!t]
\centering
\captionsetup{justification=centering}
\includegraphics[width=3in]{figures/lr20g_pr_km_wc.png}
\caption{Execution Time Prediction for Logistic Regression Job Interfered with Other Three Jobs}
\label{lr20gprkmwc}
\end{figure}
\begin{figure}[!t]
\centering
\captionsetup{justification=centering}
\includegraphics[width=3in]{figures/wc20g_pr_km_lr.png}
\caption{Execution Time Prediction for WordCount Job Interfered with Other Three Jobs}
\label{wc20gprkmlr}
\end{figure}



\begin{table*}[!t]
\renewcommand{\arraystretch}{1.3}
\caption {Prediction Accuracy for Interference Among Different Jobs Starting at Different Times}
\label{table_differentjob_differentstarttime}
\centering
\begin{tabular}{c|c|c|c|c}
\hline

\bfseries Run & \bfseries JobName & \bfseries Starting Time(s) & \bfseries First Stage & \bfseries Whole Job\\
\hline\hline
Scenario - I & PR & 0 & 0.91 & 0.81\\
& KM & 38 & 0.99 & 0.94\\
& LR & 26 & 0.94 & 0.94 \\
& WC & 78 & 0.83 & 0.82\\
\hline
Scenario - II & PR & 91 & 0.90 & 0.82\\
& KM & 0 & 0.79 & 0.77\\
& LR & 48 & 0.87 & 0.88\\
& WC & 53 & 0.99 & 0.93\\
\hline
Scenario - III & PR & 20 & 0.99 & 0.90\\
& KM & 87 & 0.98 & 0.91\\
& LR & 0 & 0.84 & 0.85\\
& WC & 48 & 0.98 & 0.91\\
\hline
Scenario - IV & PR & 77 & 0.93 & 0.85\\
& KM & 25 & 0.72 & 0.71\\
& LR & 86 & 0.99 & 0.99 \\
& WC & 0 & 0.99 & 0.93\\
\hline
\end{tabular}
\end{table*} 



% Run B

\begin{table*}[!htb]
\renewcommand{\arraystretch}{1.3}
\caption{Execution Time Prediction for PageRank Job in Scenario - I}
\label{bpr}
\centering
\begin{tabular}{l|r|r|r|r|r|r}
\hline
\bfseries StageNo & \bfseries 1 & \bfseries 2 & \bfseries 3 & \bfseries 4 & \bfseries 5 & \bfseries 6 \\
\hline \hline
Actual Time (s)
&575.3
&36.4
&33.7
&24.1
&22.5
&55.8 \\
\hline
Predicted Time (s) 
&522.9
&43.0
&10.6
&5.4
&5.0
&28.3 \\
\hline
\end{tabular}
\end{table*}

\begin{table*}[!htb]
\renewcommand{\arraystretch}{1.3}
\caption{Execution Time Prediction for K-Means Job in Scenario - I}
\label{bkm}
\centering
\begin{tabular}{l|r|r|r|r|r|r|r|r|r|r|r|r}
\hline
\bfseries StageNo & \bfseries 1 & \bfseries 2 & \bfseries 3 & \bfseries 4 & \bfseries 5 & \bfseries 6 & \bfseries 7 & \bfseries 8 & \bfseries 9 & \bfseries 10 & \bfseries 11 & \bfseries 12\\
\hline \hline
Actual Time (s)
&611.8
&6.9
&21.0
&8.2
&18.0
&7.5
&17.6
&7.2
&17.5
&7.1
&17.3
&6.9 \\
\hline
Predicted Time (s) 
&610.7
&7.4
&25.2
&18.1
&22.9
&9.7
&23.6
&8.4
&23.2
&9.2
&24.8
&6.2 \\
\hline
\end{tabular}
\end{table*}

\begin{table*}[!htb]
\renewcommand{\arraystretch}{1.3}
\caption{Execution Time Prediction for Logistic Regression Job in Scenario - I}
\label{blr}
\centering
\begin{tabular}{l|r|r|r|r|r|r|r|r|r|r}
\hline
\bfseries StageNo & \bfseries 1 & \bfseries 2 & \bfseries 3 & \bfseries 4 & \bfseries 5 & \bfseries 6 & \bfseries 7 & \bfseries 8 & \bfseries 9 & \bfseries 10 \\
\hline \hline
Actual Time (s)
&576.8
&7.4
&7.3
&7.4
&7.2
&7.1
&7.8
&7.3
&7.2
&7.1 \\
\hline
Predicted Time (s) 
&617.1
&7.1
&7.2
&7.6
&6.6
&7.0
&6.5
&7.9
&6.9
&6.3
\\
\hline
\end{tabular}
\end{table*}


\begin{table*}[!htb]
\renewcommand{\arraystretch}{1.3}
\caption{Execution Time Prediction for WordCount Job in Scenario - I}
\label{bwc}
\centering
\begin{tabular}{l|r|r}
\hline
\bfseries StageNo & \bfseries 1 & \bfseries 2 \\
\hline \hline
Actual Time (s)
& 615.2
& 61.8 \\
\hline
Predicted Time (s) 
& 506.5
& 77.8 \\
\hline
\end{tabular}
\end{table*}


\nop{maifi

\begin{figure}[!t]
\centering
\captionsetup{justification=centering}
\includegraphics[width=3in]{B_pr.png}
\caption{Execution Time Prediction for PageRank Job in Scenario - I}
\label{bpr}
\end{figure}

\begin{figure}[!t]
\centering
\captionsetup{justification=centering}
\includegraphics[width=3in]{B_km.png}
\caption{Execution Time Prediction for K-Means Job in Scenario - I}
\label{bkm}
\end{figure}

\begin{figure}[!t]
\centering
\captionsetup{justification=centering}
\includegraphics[width=3in]{B_lr.png}
\caption{Execution Time Prediction for Logistic Regression Job in Scenario - I}
\label{blr}
\end{figure}

\begin{figure}[!t]
\centering
\captionsetup{justification=centering}
\includegraphics[width=3in]{B_wc.png}
\caption{Execution Time Prediction for WordCount Job in Scenario - I}
\label{bwc}
\end{figure}

}


\subsection{Interference Among Multiple Jobs Starting at Different Times}
\label{newsection}
Finally, to test the prediction accuracy of our model where different jobs may arrive and start at different times, we use the four Apache Spark jobs and input data set as before, and start them randomly at different times. To ensure that each job will interfere with at least one other job while executing, we set the starting time for each job as $startingTime \in [minstagetime / 10, \\
minstagetime / 2]$, where $minstagetime$ represents the smallest execution time for the first stage among all the jobs. In our case, $minstagetime=190 sec$, causing the starting time for different jobs to be between 19 sec and 95 sec.

\noindent
Given the above range, for evaluation, we randomly pick one job and start at time 0, and then 
set the starting time for the remaining three jobs between 19 sec and 95 sec randomly. 
We considered four scenarios where the starting job is different in each scenario. The prediction accuracy for the whole job while running four different jobs in parallel starting at different times is summarized in Table~\ref{table_differentjob_differentstarttime}. As shown in the table, in our evaluation, prediction accuracy ranges between 99\% and 71\% for the whole job, and between 99\% and 72\% for the first stage. The predicted execution time and the actual execution time for PageRank, K-Means, Logistic Regression, and WordCount under Scenario - I are shown in Table~\ref{bpr}, Table~\ref{bkm}, Table~\ref{blr}, and Table~\ref{bwc} respectively. 




\section{Discussion}
\label{discussion}
While our model can predict performance degradation due to interference among multiple Apache Spark jobs with high accuracy, we do acknowledge several limitations of our current work as follows. First, our model assumes that the stages within a job are executed sequentially and does not consider the possibility of parallel execution, which will require extending our models. Second, the model was evaluated on 6 node cluster with 4 concurrent jobs, which is smaller compared to real-life clusters. Furthermore, the value $\beta_{n}$ depends on the number of concurrent jobs and needs to be calculated (once) as the number of concurrent jobs grows in a system. However, we strongly believe that our modeling framework will work well once the parameters are estimated for different cluster size and number of concurrent jobs in a system, which we plan to investigate in our future work. 







\section{Conclusion}
\label{conclusion}
In this paper, to predict the execution time of Apache Spark jobs interfered with other jobs, we develop an interference model. This model combines the execution information and resource consumption profile for each stage of Apache Spark jobs to calculate the slowdown ratio resulting from the interference, and then predicts the execution time when interfered with other jobs. Furthermore, an interference aware job scheduling algorithm leveraging the analytical framework is designed for Apache Spark platform. The developed models and the algorithm are evaluated using four real-life applications (e.g., Page rank , K-means, Logistic regression, Word count) on a 6 node cluster while running up to four jobs concurrently. Experimental results demonstrate that our framework can achieve high prediction accuracy and reduces average execution time for individual jobs significantly, thereby improving the system utilization. 






\section{Acknowledgement} 
This work is supported by the AFOSR under Grant No.
\linebreak FA 9550-15-1-0184. Any opinions, findings, and conclusions
or recommendations expressed in this material are those of
the authors and do not necessarily reflect the views of the
funding agency.


\bibliographystyle{IEEEtran}
\bibliography{paper}


\end{document}


